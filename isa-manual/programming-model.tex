% -*- mode: latex; tab-width: 2; indent-tabs-mode: nil; -*-
%------------------------------------------------------------------------------
% MRISC32 ISA Manual - Programming model.
%
% This work is licensed under the Creative Commons Attribution-ShareAlike 4.0
% International License. To view a copy of this license, visit
% http://creativecommons.org/licenses/by-sa/4.0/ or send a letter to
% Creative Commons, PO Box 1866, Mountain View, CA 94042, USA.
%------------------------------------------------------------------------------

\chapter{Programming model}

\section{Scalar registers}

There are 32 scalar registers, each 32 bits wide.

\begin{bytefield}{32}
  \bitheader{0,31} \\
  \wordbox{1}{Z (S0)} \\
  \wordbox{1}{S1} \\
  \wordbox{1}{S2} \\
  \wordbox[]{1}{$\vdots$} \\[1ex]
  \wordbox{1}{S24} \\
  \wordbox{1}{S25} \\
  \wordbox{1}{FP (S26)} \\
  \wordbox{1}{TP (S27)} \\
  \wordbox{1}{SP (S28)} \\
  \wordbox{1}{VL (S29)} \\
  \wordbox{1}{LR (S30)} \\
  \wordbox{1}{PC (S31)}
\end{bytefield}

\subsection{The Z register}

Z is a read-only register that is always zero. Writing to the Z register has no
effect.

\subsection{The VL register}

VL is the vector length register, which defines the length of vector
operations.

\todo{Describe how the VL register affects vector operations.}

\subsection{The LR register}

LR is the link register, which contains the return address for subroutines.

\subsection{The PC register}

The PC is a read-only register that contains the memory address of the current
instruction. Writing to the PC register has no effect.

\subsection{FP, TP and SP}

The scalar registers FP, TP and SP are aliases for S26, S27 and S28,
respectively. They have no special meaning in hardware, but it is recommended
that they are used as follows:

\begin{tabular}{|l|l|}
  \hline
  \textbf{Name} & \textbf{Description} \\
  \hline
  FP & Frame pointer \\
  \hline
  TP & Thread pointer (for thread local storage) \\
  \hline
  SP & Stack pointer \\
  \hline
\end{tabular}

\section{Vector registers}

There are 32 vector registers: VZ, V1, V2, \dots, V30, V31.

Each register, V$k$, consists of $N$ 32-bit elements, where $N$ is
implementation defined ($N$ must be a power of two, and at least 16):

\begin{bytefield}{32}
  \bitheader{0,31} \\
  \wordbox{1}{V$k[0]$} \\
  \wordbox{1}{V$k[1]$} \\
  \wordbox{1}{V$k[2]$} \\
  \wordbox{1}{V$k[3]$} \\
  \wordbox{1}{V$k[4]$} \\
  \wordbox[]{1}{$\vdots$} \\[1ex]
  \wordbox{1}{V$k[N-2]$} \\
  \wordbox{1}{V$k[N-1]$}
\end{bytefield}

\subsection{The VZ register}

VZ is a read-only register with all vector elements set to zero. Writing to the
VZ register has no effect.

\section{Memory addressing}

\tbd

\section{Vector operation}

\tbd

\section{Packed data operation}

\tbd

\section{Exceptions}

\tbd
