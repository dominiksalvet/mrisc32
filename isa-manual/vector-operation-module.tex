% -*- mode: latex; tab-width: 2; indent-tabs-mode: nil; -*-
%------------------------------------------------------------------------------
% MRISC32 ISA Manual - Vector operation module.
%
% This work is licensed under the Creative Commons Attribution-ShareAlike 4.0
% International License. To view a copy of this license, visit
% http://creativecommons.org/licenses/by-sa/4.0/ or send a letter to
% Creative Commons, PO Box 1866, Mountain View, CA 94042, USA.
%------------------------------------------------------------------------------

\chapter{Vector operation module}

The vector operation module adds facilities for vector processing. A set of
vector registers is added, and most instructions are extended to support
processing of vector registers.

\section{Vector registers}

There are 32 vector registers:

\begin{tabular}{|l|l|}
  \hline
  \textbf{Vector reg. no} & \textbf{Name} \\
  \hline
  0 & VZ \\
  \hline
  1 & V1 \\
  \hline
  2 & V2 \\
  \hline
  3 & V3 \\
  \hline
  4 & V4 \\
  \hline
  \multicolumn{2}{c}{$\vdots$} \\
  \hline
  30 & V30 \\
  \hline
  31 & V31 \\
  \hline
\end{tabular}

Each register, V$k$, consists of $N$ 32-bit elements, where $N$ is
implementation defined ($N$ must be a power of two, and at least 16):

\begin{bytefield}{32}
  \bitheader{0,31} \\
  \wordbox{1}{V$k[0]$} \\
  \wordbox{1}{V$k[1]$} \\
  \wordbox{1}{V$k[2]$} \\
  \wordbox{1}{V$k[3]$} \\
  \wordbox{1}{V$k[4]$} \\
  \wordbox[]{1}{$\vdots$} \\[1ex]
  \wordbox{1}{V$k[N-2]$} \\
  \wordbox{1}{V$k[N-1]$}
\end{bytefield}

\subsection{The VZ register}

VZ is a read-only register with all vector elements set to zero. Writing to the
VZ register has no effect.

\section{Vector operation}

A vector operation is performed when a source or destination operand of an
instruction is a vector register.

\begin{todobox}
  Describe how vector operations process elements of vector registers.
\end{todobox}

\section{Vector length}

\begin{todobox}
  Describe how VL affects vector operations.

  Describe how to obtain the maximum vector length for an implementation.
\end{todobox}

\section{Folding}

Horizontal vector operations (e.g. sum and min/max) are supported by repeated
folding, where the upper half of one vector source operand is combined with the
lower half of another vector source operand.

\begin{todobox}
  Describe how folding works.
\end{todobox}
